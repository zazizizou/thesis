\chapter{Introduction}
	Air-water gas exchange is important in many contexts in nature and engineering. Climate change is probably one of the most prevailing topics involving air-water exchange between the atmosphere and ocean. 
	
	The transfer rate of most gases between the atmosphere and ocean is controlled by processes just beneath the water surface. When the water surface is highly turbulent, gases can be more rapidly transferred toward or away from the surface. In particular, turbulent regions can generate breaking waves which in turn create bubbles that trap air from the atmosphere into the ocean. These bubbles enlarge the air-water interface due to their additional surface under water. However, this does not necessarily mean that gas exchanges will occur for the whole bubble lifetime. In fact, smaller bubbles can reach equilibrium with the surrounding water and stop exchanging gases. Therefore, determining bubble radii is crucial to properly model bubble induced gas exchange.
	
	The goal of this work is to develop a measurement technique that determines the bubble spectrum in a given measurement volume, i.e. determine bubble concentrations as a function of their radii in a well determined volume. 
	
	In this work, we first discuss the relevant theoretical basics in chapter \ref{theory} starting with the physics behind capturing bubble images, followed by basics in image processing. Next, we briefly discuss related measurement techniques in chapter \ref{related_work} and their importance to our work. In chapter \ref{experimental_setup}, we describe our experimental setup in detail. Our proposed algorithm that analyses bubble images is described thoroughly in chapter \ref{the_algorithm}. 
\chapter{Introduction}
	Air-water gas exchange is important in many contexts in nature and engineering. Climate change is probably one of the most prevailing topics involving air-water exchange between the atmosphere and ocean \citep{NIWA}.
	
	The transfer rate of most gases between the atmosphere and ocean is controlled by processes just beneath the water surface. Wind blowing over the sea is the main driving force for all relevant physical processes. 
	When the water surface is highly turbulent, gases can be more rapidly transferred toward or away from the surface. In particular, turbulent regions can generate breaking waves which in turn create bubbles that trap air from the atmosphere into the ocean \citep{Terry}. These bubbles enlarge the air-water interface due to their additional surface under water. However, this does not necessarily mean that gas exchanges will occur for the whole bubble lifetime. In fact, smaller bubbles can reach equilibrium with the surrounding water and stop exchanging gases \citep{MischlerDiss}. Therefore, determining bubble radii is crucial to properly model bubble induced gas exchange.
	
	Previously developed bubble measuring methods in wind wave facilities include light scattering based methods \citep{jaehne1984}, depth from focus methods \citep{geissler_1995} and more recently, light field methods \citep{MischlerDiss}. Other in situ methods such as \citet{Al-Lashi2016} and \citet{Leifer2003} introduced instruments for imaging bubbles within breaking waves. These methods rely on thresholding and edge detection algorithms to produce binary images. Further techniques such as Hough transform \citep{Hough1972} are applied to extract bubble distributions. Many of these methods, however, are very sensitive to background illumination, which either reduces their accuracy \citep{Zhong2016} or demands more manual preprocessing.  
	
	The goal of this work is to develop a measurement technique that determines the bubble spectrum in a given measurement volume, i.e. determine bubble concentrations as a function of their radii in a well determined volume. In particular, this work uses recent developments in object detection algorithms, while relying on machine learning for bubble classification in order to develop an accurate and robust bubble measurement technique. 
	
	In this work, first the relevant theoretical basics are discussed in chapter \ref{theory} starting with the physics behind capturing bubble images, followed by basics in image processing. Next, relevant related measurement techniques and their importance to this work are briefly discussed in chapter \ref{related_work}. Chapter \ref{experimental_setup} describes the experimental setup in detail. The proposed algorithm that analyses bubble images is discussed thoroughly in chapter \ref{the_algorithm}. Finally, the performance of the proposed algorithms is discussed in chapter \ref{results}.
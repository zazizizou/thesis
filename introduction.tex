\chapter{Introduction}
	Air-water gas exchange is important in many contexts in nature and engineering. Climate change is one of the most prevailing topics involving air-water exchange between the atmosphere and ocean \citep{MischlerDiss}.
	
	The transfer rate of most gases between the atmosphere and ocean is controlled by processes just beneath the water surface. Wind blowing over the sea is the main driving force for all relevant physical processes. 
	When the water surface is highly turbulent, gases can be more rapidly transferred toward or away from the surface. In particular, turbulent regions can generate breaking waves which in turn create bubbles that trap air from the atmosphere into the ocean \citep{Terry}. These bubbles enlarge the air-water interface due to their additional surface under water. However, this does not necessarily mean that gas exchanges will occur for the whole bubble lifetime. In fact, smaller bubbles can reach equilibrium with the surrounding water and stop exchanging gases \citep{MischlerDiss}. Therefore, determining bubble radii as well as their distribution is crucial to properly model bubble induced gas exchange.
	
	Previously developed bubble measuring methods in wind wave facilities include light scattering based methods \citep{jaehne1984}, depth from focus methods \citep{geissler_1995} and more recently, light field methods \citep{MischlerDiss}. The idea behind these methods is to light air bubbles from behind, such that light gets scattered away by bubbles and only background light reaches the camera facing the light source. This produces images containing dark disks that can be analyzed using a circle fit. Although this approach yields good results for low bubble concentrations, it does not perform well for high bubble concentrations, partly because not enough light can reach the camera if bubbles are overlapping or too close to each other. 
	
	Other in situ methods such as \citet{Al-Lashi2016} and \citet{Leifer2003} introduced instruments for imaging bubbles within breaking waves. These methods light air bubbles from below or above and measure the light reflected by the bubbles to detect air bubbles and estimate their radii. Thresholding and edge detection algorithms are then applied to produce binary images. Further techniques such as the Hough transform \citep{Hough1972} are used to extract bubble distributions. Many of these methods, however, are very sensitive to background illumination, which either reduces their accuracy \citep{Zhong2016} or demands more manual pre-processing.
	
	\newpage
	The goal of this work is to develop an imaging system and an algorithm that determine the bubble spectrum in a given measurement volume, i.e.\ determine bubble concentrations as a function of their radii in a well determined volume. In particular, the developed method aims to measure bubbles close to the water surface and perform well for low and high bubble concentrations.  

	In this work, first the relevant theoretical basics are discussed in chapter \ref{theory} starting with the physics behind capturing bubble images, followed by the basics in image processing. Next, relevant bubble imaging methods and their importance to this work are briefly discussed in chapter \ref{related_work}. Chapter \ref{experimental_setup} describes the experimental setup of the proposed method in detail, and the proposed algorithm that analyses bubble images is discussed thoroughly in chapter \ref{the_algorithm}. Finally, the performance of the proposed algorithms is discussed in chapter \ref{results}.
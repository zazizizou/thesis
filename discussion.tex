\chapter{Conclusion}\label{discussion}
A bubble measuring technique has been developed with special emphasis on the ability to measure bubbles close to the water surface and developing a robust bubble detection algorithm. 

\textbf{Measurement setup}: Artificially generated bubbles were lit from below, which resulted in images where bubbles are characterized by two peaks for a low bubble concentrations, and images where bubbles had a more apparent curvature around the bubble edge for high bubble concentrations. Keeping the angle between the camera and the light source at 90$^\circ$ insured a consistent pattern across the different used water tanks. The drawback of this method is that only medium to large bubbles can be detected since only total reflections from the bubble's lower edge are visible. 

\textbf{Algorithm \ref{algo:bubbleNet}}: This algorithm uses a deep learning approach. So generating a large amount of annotated training data was necessary. This was achieved through simulation as well as through annotating measurement results with a logistic regression classifier, i.e. a simpler classifier was optimized to detect true positive bubble instances only, which were then used to train the convolutional neural network. This algorithm performed well when measured by the mAP@0.5IoU criteria. Also, the variation within the validation dataset was strong, therefore it has been verified that this algorithm is robust. Using a deep learning approach however comes with an inherent limit, which is the lack of interpretability of the model. 

\textbf{Algorithm \ref{algo:bubbleCurves}}: Images with large bubble concentrations show bubbles with a nicely recognizable curvature. This observation was used in this algorithm to classify bubbles and determine their radii. For classification, features that extract orientation and edge's concavity from a potential bubble were used. The radius is then determined from the bubble's lower half. This algorithm performed slightly worse than algorithm \ref{algo:bubbleNet}. The validation dataset was also not as varied as the one used for the previous algorithm. Therefore this algorithm is less robust against large changes in camera settings and lighting conditions. The advantage however, is that features can be easily interpretable since they were chosen manually. 

\textbf{Calibration}: It has been shown that the width of the brightest peak at the bubble's lower edge can be used to determine the bubble's depth, i.e. location in the third dimension. As for the radius calibration, an additional light source was used to capture bubbles as dark circles, track them, compute their radii and then find a function that describes their radii with the ones computed with algorithms \ref{algo:bubbleNet} and \ref{algo:bubbleCurves}. In both calibration setups real bubbles were used, i.e. no calibration target that emulates bubbles was needed, which contributed positively to calibration accuracy. 

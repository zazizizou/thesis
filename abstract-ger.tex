Der Beitrag der durch brechende Wellen entstehende Luftblasen zum Gasaustausch zwischen Ozean und Atmosph\"are ist umstritten. Deshalb ist es wichtig, sowohl die Konzentration als auch die Verteilung dieser Luftblasen abzusch\"atzen, um eine kr\"aftige Aussage über ihren Beitrag zum Gasaustausch treffen zu k\"onnen. 

In dieser Arbeit wurde eine neue Messmethode sowie bildverarbeitungsbasierte Algorithmen entwickelt, mit deren Hilfe Blasen und deren Radien zuverlässig und automatisiert in einem Wind-Wellenkanal detektiert werden k\"onnen. Eine Beleuchtung der Luftblasen von Unten erm\"oglicht die Messung der Blasen, die sich in der Nähe von der Wasseroberfläche befinden. 

Zur Auswertung wurden zwei Algorithmen entwickelt. Der erste basiert sich auf einen Deep-Learning Ansatz und eignet sich gut f\"ur Messungen mit kleinen Blasenkonzentrationen, während der zweite klassische Machine Learning- und Bildverarbeitungsmethoden zur Blasendetektion bei hohen Blasenkonzentrationen verwendet. Die Lokaliesierung der Blasen in der dritten Dimension konnte durch die Kalibrierung der Tiefensch"arfe bestimmt werden.  
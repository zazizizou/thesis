The contribution of air bubbles emerging from breaking waves to the gas exchange between atmosphere and ocean is still a controversial topic. Therefore, it is necessary to estimate bubble concentrations as well as their distribution in order to determine their importance to the gas exchange. 

In this thesis, a new measuring technique has been developed, where bubbles and their radii can be automatically and reliably detected in a wind-wave facility. 
The experimental setup relies on lighting bubbles from below allows the detection of bubbles that are very close to the water surface. 

Two separate algorithms have been developed to solve the bubble detection problem. The first is a deep learning approach that works best for measurements with a low bubble concentration. The second one uses a classical machine learning and image processing approach to detect bubbles and estimate their radii. It was designed to perform well for images with high bubble concentrations. Determining bubble depths, i.e.\ the distance to the focal plane was also made possible using a depth from focus technique. 